\documentclass[a4paper,12pt]{article}
\usepackage[utf8]{inputenc}
\usepackage{graphicx}
\usepackage{geometry}

\usepackage{setspace}
\usepackage{parskip}

\begin{document}

\begin{titlepage}
    \centering
    \begin{flushright}
        \includegraphics[width=0.5\textwidth]{images/HFU-Logo.png}
    \end{flushright}
    
    \vspace{4cm}
    
    {\Large \textbf{Dokumentation}}\\[0.5cm]
    {\huge \textbf{Digitale Influencer}}\\[1.5cm]
    
    \textbf{Semesterprojekt Sommersemester 2025}
    
    \vfill
    
    \begin{flushleft}
    \textbf{Referent:} \\
    Herr Prof. Dr. Pascal Laube\\[2.5cm]
    
    \textbf{Autoren:} \\
    Marvin Jonas Kern \\
    Kevin Maisler \\
    Julia Maria Riebel \\
    Bünyamin Sener \\
    \end{flushleft}
    
    \vfill
    \begin{center}
        {\large \today}
    \end{center}
\end{titlepage}

\renewcommand*\contentsname{Inhaltsverzeichnis}
\tableofcontents
\newpage

\section{Einleitung}
Im Sommersemester 2025 wurde im Rahmen der Studiengänge Angewandte Informatik und IT-Produktmanagement an der Hochschule Furtwangen ein Projekt mit dem Titel „Digitale Influencer – Aufbau einer virtuellen Social Media Präsenz“ durchgeführt. Unter der Betreuung von Prof. Dr. Pascal Laube arbeitete ein vierköpfiges Studierendenteam daran, innerhalb eines Semesters eine vollständig digitale Influencer-Persona zu konzipieren, technisch umzusetzen und auf relevanten sozialen Netzwerken sichtbar zu machen und zu etablieren. \\\\
Ziel dieses Projekts war es, mithilfe moderner KI-Tools eine virtuelle InfluencerIn zu erschaffen, die durch automatisiert erstellte Foto- und Videoinhalte eine möglichst große Reichweite und Followerzahl generieren kann. Dabei lag der Fokus insbesondere auf der Entwicklung einer einzigartigen digitalen Identität, der strategischen Content-Planung sowie der algorithmischen Optimierung der Inhalte und Interaktionen auf Plattformen wie Instagram und TikTok.\\\\
Die Projektarbeit umfasste neben der technischen Umsetzung auch kreative und analytische Aufgaben: von der Recherche aktueller Social-Media-Trends über die KI-basierte Content-Produktion bis hin zur kontinuierlichen Auswertung von \\ Engagement-Daten zur Strategieoptimierung. Die Herausforderung bestand darin, technologische Innovation mit kreativem Storytelling zu verbinden – eine interdisziplinäre Aufgabe, die ein hohes Maß an Teamarbeit, Experimentierfreude und digitale Affinität erforderte.

\newpage

\section{Vorbereitungen}

In diesem Kapitel werden die allgemeinen Vorbereitungen für dieses Projekt behandelt. 

\subsection{Rechtliche Ausgangslage}

\subsection{Marktanalyse}
\subsection{Toolrecherche}

\section{Auswahl der Technologien}
\subsection{Python}
\subsection{Github}
\subsection{Ollama}
\subsection{Fooocus}
\subsection{Meta-API}


\section{KI Influencer}

Im Rahmen unseres KI-Influencer-Projekts fiel die Entscheidung bewusst auf die Konzeption eines digitalen Mentors, der sich durch Tiefe, Klarheit und eine inspirierende Ausstrahlung von der Masse abheben soll. Während viele virtuelle Influencer stark auf Ästhetik oder Mode fokussiert sind, war es unser Ziel, eine Figur mit inhaltlicher Tiefe und echtem Mehrwert zu schaffen – einen Charakter, der wie ein mentaler Coach oder Berater in digitalen Zeiten agiert.

So entstand \textbf{Elias Lev}, ein KI-generierter Influencer im Alter von 31 Jahren. Sein Auftreten ist geprägt von einem \textit{urban-clean Look}, einer ruhigen, leicht stoischen Ausstrahlung und einer tech-affinen Note. Seine Sprache ist klar, direkt und inspirierend – ohne in Motivationsfloskeln zu verfallen. Elias soll vor allem Menschen zwischen \textbf{20 und 35 Jahren} ansprechen, die sich für \textit{Selbstoptimierung, mentale Klarheit, emotionale Intelligenz und die Zukunft von KI im Alltag} interessieren.

Die Inhalte, die Elias teilt, orientieren sich an etablierten Social-Media-Formaten und verbinden sie mit einem reflektierten Ton:

\begin{itemize}
    \item \textbf{Mindset Monday}: Kurze Reels mit Zitaten und Voiceover, die zum Wochenstart Denkanstöße geben
    \item Antworten auf Community-Kommentare in einem ruhigen, erklärenden Stil
    \item Beiträge zu Routinen, Umgang mit Druck und bewusster Lebensgestaltung
\end{itemize}

Sein Erscheinungsbild ist bewusst \textit{strukturiert und leicht distanziert}, um den Eindruck eines Thought-Leaders zu unterstützen – jemand, der weniger „Freund“, sondern vielmehr ein ruhiger Gegenpol zum hektischen Informationsrauschen digitaler Plattformen ist.

Um Elias visuell zu erschaffen, haben wir mithilfe von \textbf{Fooocus} ein erstes Bild generiert. Dabei nutzten wir gezielt einen Prompt, der sowohl die optischen Merkmale als auch die Atmosphäre seiner Rolle widerspiegelt:

\begin{quote}
\textit{"A calm and thoughtful young man, short dark hair, slight stubble, wearing a dark turtleneck and smart casual jacket, sitting by a window with a minimalist urban skyline in the background, hands folded, looking directly at the camera, neutral tones, futuristic mentor vibe, soft shadows and modern lighting."}
\end{quote}

Dieses Bild diente als Grundlage für die erste visuelle Iteration von Elias und spiegelt seine Funktion als \textbf{zukünftiger Mentor in einer KI-geprägten Welt} wider.


\section{Automatisierung}
\subsection{Zitatgenerierung}
\subsection{Bildgenerierung}
\subsection{Automatisches Posten}
\subsection{Interaktion mit Community}

\section{Follower Analyse}

\section{Herausforderungen}
\subsection{Datenschutz}
\subsection{Meta-API}
\subsection{Videogenerierung}

\section{Fazit und Zukunftsaussichten}

\end{document}
