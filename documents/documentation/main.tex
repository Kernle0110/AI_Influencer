\documentclass[a4paper,12pt]{article}
\usepackage[utf8]{inputenc}
\usepackage{graphicx}
\usepackage{geometry}

\usepackage{setspace}
\usepackage{parskip}

\begin{document}

\begin{titlepage}
    \centering
    \begin{flushright}
        \includegraphics[width=0.5\textwidth]{images/HFU-Logo.png}
    \end{flushright}
    
    \vspace{4cm}
    
    {\Large \textbf{Dokumentation}}\\[0.5cm]
    {\huge \textbf{Digitale Influencer}}\\[1.5cm]
    
    \textbf{Semesterprojekt Sommersemester 2025}
    
    \vfill
    
    \begin{flushleft}
    \textbf{Referent:} \\
    Herr Prof. Dr. Pascal Laube\\[2.5cm]
    
    \textbf{Autoren:} \\
    Marvin Jonas Kern \\
    Kevin Maisler \\
    Julia Maria Riebel \\
    Bünyamin Sener \\
    \end{flushleft}
    
    \vfill
    \begin{center}
        {\large \today}
    \end{center}
\end{titlepage}

\renewcommand*\contentsname{Inhaltsverzeichnis}
\tableofcontents
\newpage

\section{Einleitung}
\text{Im Sommersemester 2025 wurde im Rahmen der Studiengänge Angewandte Informatik und IT-Produktmanagement an der Hochschule Furtwangen ein Projekt mit dem Titel „Digitale Influencer – Aufbau einer virtuellen Social Media Präsenz“ durchgeführt. Unter der Betreuung von Prof. Dr. Pascal Laube arbeitete ein vierköpfiges Studierendenteam daran, innerhalb eines Semesters eine vollständig digitale Influencer-Persona zu konzipieren, technisch umzusetzen und auf relevanten sozialen Netzwerken sichtbar zu machen und zu etablieren. \\\\
Ziel dieses Projekts war es, mithilfe moderner KI-Tools eine virtuelle InfluencerIn zu erschaffen, die durch automatisiert erstellte Foto- und Videoinhalte eine möglichst große Reichweite und Followerzahl generieren kann. Dabei lag der Fokus insbesondere auf der Entwicklung einer einzigartigen digitalen Identität, der strategischen Content-Planung sowie der algorithmischen Optimierung der Inhalte und Interaktionen auf Plattformen wie Instagram und TikTok.\\\\
Die Projektarbeit umfasste neben der technischen Umsetzung auch kreative und analytische Aufgaben: von der Recherche aktueller Social-Media-Trends über die KI-basierte Content-Produktion bis hin zur kontinuierlichen Auswertung von \\ Engagement-Daten zur Strategieoptimierung. Die Herausforderung bestand darin, technologische Innovation mit kreativem Storytelling zu verbinden – eine interdisziplinäre Aufgabe, die ein hohes Maß an Teamarbeit, Experimentierfreude und digitale Affinität erforderte.}

\newpage

\end{document}
